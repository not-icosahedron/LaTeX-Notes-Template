%----------------------------------------------------------------------------------------
%	PACKAGES AND OTHER DOCUMENT CONFIGURATIONS
%----------------------------------------------------------------------------------------

\usepackage{amsmath,amsfonts,stmaryrd,amssymb} % Math packages

\usepackage{enumerate} % Custom item numbers for enumerations

\usepackage[ruled]{algorithm2e} % Algorithms

\usepackage[framemethod=tikz]{mdframed} % Allows defining custom boxed/framed environments

\usepackage{listings} % File listings, with syntax highlighting
\lstset{
	basicstyle=\ttfamily, % Typeset listings in monospace font
}

%----------------------------------------------------------------------------------------
%	DOCUMENT MARGINS
%----------------------------------------------------------------------------------------

\usepackage[asymmetric]{geometry} % Required for adjusting page dimensions and margins

\geometry{
	paper=a4paper, % Paper size, change to letterpaper for US letter size
	top=1.2cm, % Top margin
	bottom=2.5cm, % Bottom margin
	left=1.2cm, % Left margin
	right=1.2cm, % Right margin
	headheight=5pt, % Header height
	footskip=1.0cm, % Space from the bottom margin to the baseline of the footer
	headsep=0.3cm % Space from the top margin to the baseline of the header
	%showframe, % Uncomment to show how the type block is set on the page
   % bindingoffset=0.7cm
}

%----------------------------------------------------------------------------------------
%	FONTS
%----------------------------------------------------------------------------------------

\usepackage[utf8]{inputenc} % Required for inputting international characters
\usepackage[T1]{fontenc} % Output font encoding for international characters

\usepackage{fontspec} % Required for specifying custom fonts

\defaultfontfeatures{Ligatures=TeX} % To support LaTeX ligatures (`` and --)
\defaultfontfeatures{Path=Fonts/} % Specify the location of font files

% Set the main sans font
\setmainfont{Merriweather}[
	UprightFont=*-Regular.otf,
	BoldFont=*-Bold.otf,
	ItalicFont=*-Italic.otf,
	BoldItalicFont=*-BoldItalic.otf
]

\newfontface\latoblack{Lato-Black.ttf} % Define a command for a single custom font
%----------------------------------------------------------------------------------------
%	COMMAND LINE ENVIRONMENT
%----------------------------------------------------------------------------------------

% Usage:
% \begin{commandline}
%	\begin{verbatim}
%		$ ls
%		
%		Applications	Desktop	...
%	\end{verbatim}
% \end{commandline}

\mdfdefinestyle{commandline}{
	leftmargin=10pt,
	rightmargin=10pt,
	innerleftmargin=15pt,
	middlelinecolor=black!50!white,
	middlelinewidth=2pt,
	frametitlerule=false,
	backgroundcolor=black!5!white,
	frametitle={example commands},
	frametitlefont={\normalfont\sffamily\color{white}\hspace{-1em}},
	frametitlebackgroundcolor=black!50!white,
	nobreak,
}

% Define a custom environment for command-line snapshots
\newenvironment{commandline}{
	\medskip
	\begin{mdframed}[style=commandline]
}{
	\end{mdframed}
	\medskip
}

%----------------------------------------------------------------------------------------
%	FILE CONTENTS ENVIRONMENT
%----------------------------------------------------------------------------------------

% Usage:
% \begin{file}[optional filename, defaults to "File"]
%	File contents, for example, with a listings environment
% \end{file}

\mdfdefinestyle{file}{
	innertopmargin=1.6\baselineskip,
	innerbottommargin=0.8\baselineskip,
	topline=false, bottomline=false,
	leftline=false, rightline=false,
	leftmargin=2cm,
	rightmargin=2cm,
	singleextra={%
		\draw[fill=black!10!white](P)++(0,-1.2em)rectangle(P-|O);
		\node[anchor=north west]
		at(P-|O){\ttfamily\mdfilename};
		%
		\def\l{3em}
		\draw(O-|P)++(-\l,0)--++(\l,\l)--(P)--(P-|O)--(O)--cycle;
		\draw(O-|P)++(-\l,0)--++(0,\l)--++(\l,0);
	},
	nobreak,
}

% Define a custom environment for file contents
\newenvironment{file}[1][File]{ % Set the default filename to "File"
	\medskip
	\newcommand{\mdfilename}{#1}
	\begin{mdframed}[style=file]
}{
	\end{mdframed}
	\medskip
}

%----------------------------------------------------------------------------------------
%	NUMBERED QUESTIONS ENVIRONMENT
%----------------------------------------------------------------------------------------

% Usage:
% \begin{question}[optional title]
%	Question contents
% \end{question}

\mdfdefinestyle{question}{
	innertopmargin=1.2\baselineskip,
	innerbottommargin=0.8\baselineskip,
	roundcorner=5pt,
	nobreak,
	singleextra={%
		\draw(P-|O)node[xshift=1em,anchor=west,fill=white,draw,rounded corners=5pt]{%
		Question \theQuestion\questionTitle};
	},
}

\newcounter{Question} % Stores the current question number that gets iterated with each new question

% Define a custom environment for numbered questions
\newenvironment{question}[1][\unskip]{
	\bigskip
	\stepcounter{Question}
	\newcommand{\questionTitle}{~#1}
	\begin{mdframed}[style=question]
}{
	\end{mdframed}
	\medskip
}

%----------------------------------------------------------------------------------------
%	WARNING TEXT ENVIRONMENT
%----------------------------------------------------------------------------------------

% Usage:
% \begin{warn}[optional title, defaults to "Warning:"]
%	Contents
% \end{warn}

\mdfdefinestyle{warning}{
	topline=false, bottomline=false,
	leftline=false, rightline=false,
	nobreak,
	singleextra={%
		\draw(P-|O)++(-0.5em,0)node(tmp1){};
		\draw(P-|O)++(0.5em,0)node(tmp2){};
		\fill[black,rotate around={45:(P-|O)}](tmp1)rectangle(tmp2);
		\node at(P-|O){\color{white}\scriptsize\textbf !};
		\draw[very thick](P-|O)++(0,-1em)--(O);%--(O-|P);
	}
}

% Define a custom environment for warning text
\newenvironment{warn}[1][Note:]{ % Set the default warning to "Note:"
	\medskip
	\begin{mdframed}[style=warning]
		\noindent{\textbf{#1}}
}{
	\end{mdframed}
}

%----------------------------------------------------------------------------------------
%	ENTRY LIST
%----------------------------------------------------------------------------------------

\usepackage{longtable} % Required for tables that span multiple pages
\setlength{\LTpre}{0pt} % Remove default whitespace before longtable
\setlength{\LTpost}{0pt} % Remove default whitespace after longtable

\setlength{\tabcolsep}{0pt} % No spacing between table columns

% Environment to hold a new list of entries
\newenvironment{entrylist}{
	\begin{longtable}[H]{l l}
}{
	\end{longtable}
}

\newcommand{\entry}[4]{% First argument for the leftmost date(s) text, second is for the bold entry heading, third is for the bold right-aligned entry qualifier and the fourth is for the entry description
	\parbox[t]{0.175\textwidth}{% 17.5% of the text width of the page
		#1 % Leftmost entry date(s) text
	}%
	&\parbox[t]{0.825\textwidth}{% 82.5% of the text width of the page
		\textbf{#2}% Entry heading text
		\hfill% Horizontal whitespace
		{\footnotesize \textbf{\textcolor{black}{#3}}}\\% Right-aligned entry qualifier text
		#4 % Entry description text
	}\\\\}

% Command to output a separator slash between lists, e.g. '  /  '
\newcommand{\slashsep}{\hspace{3mm}/\hspace{3mm}}

%----------------------------------------------------------------------------------------
%	INFORMATION ENVIRONMENT
%----------------------------------------------------------------------------------------

% Usage:
% \begin{info}[optional title, defaults to "Info:"]
% 	contents
% 	\end{info}

\mdfdefinestyle{info}{%
	topline=false, bottomline=false,
	leftline=false, rightline=false,
	nobreak,
	singleextra={%
		\fill[black](P-|O)circle[radius=0.4em];
		\node at(P-|O){\color{white}\scriptsize\textbf i};
		\draw[very thick](P-|O)++(0,-0.8em)--(O);%--(O-|P);
	}
}

% Define a custom environment for information
\newenvironment{info}[1][Info:]{ % Set the default title to "Info:"
	\medskip
	\begin{mdframed}[style=info]
		\noindent{\textbf{#1}}
}{
	\end{mdframed}
}

%-------------STICKY NOTES----------
\usepackage{tikz, adjustbox}
\usepackage[most]{tcolorbox}
\usepackage{xcolor}
\usepackage{wrapfig}

%           Defining colors for sticky notes:
%_______________________________________________________________
% Yellow:
\definecolor{BgYellow}{HTML}{FFF59C}
\definecolor{FrameYellow}{HTML}{F7A600}
% Pink:
\definecolor{BgPink}{HTML}{EF6FA7}
\definecolor{FramePink}{HTML}{E5446E}
% Green:
\definecolor{BgGreen}{HTML}{C7D92D}
\definecolor{FrameGreen}{HTML}{89B23B}
% Blue:
\definecolor{BgBlue}{HTML}{45BEE9}
\definecolor{FrameBlue}{HTML}{31A8C9}
% White:
\definecolor{BgWhite}{HTML}{D8D8D8}
\definecolor{FrameWhite}{HTML}{7F7F7F}
% Brown:
\definecolor{BgBrown}{HTML}{8E7A45}
\definecolor{FrameBrown}{HTML}{6B5B32}
%................................................................
%                   NB command:
%_______________________________________________________________
\usepackage{contour}
\newcommand{\NB}{\contour{black}{\textbf{{\large\sffamily\color{red}NB}}}\textbf{\large\sffamily: }}
%................................................................
%
%               Defining Sticky note boxes:
%_______________________________________________________________
% Yellow Sticky Note (YStkyNote):
\newtcolorbox{YStkyNote}[1][]{%
    enhanced,
    before skip=2mm,after skip=2mm, 
    width=0.44\textwidth, % width of the sticky note
    boxrule=0.2mm,
    colback=BgYellow, colframe=FrameYellow, % Colors
    attach boxed title to top left={xshift=0cm,yshift*=0mm-\tcboxedtitleheight},
    varwidth boxed title*=-3cm,
    % The titlebox:
    boxed title style={frame code={%
        \path[left color=FrameYellow,right color=FrameYellow,
        middle color=FrameYellow]
        ([xshift=-0mm]frame.north west) -- ([xshift=0mm]frame.north east)
        [rounded corners=0mm]-- ([xshift=0mm,yshift=0mm]frame.north east)
        -- (frame.south east) -- (frame.south west)
        -- ([xshift=0mm,yshift=0mm]frame.north west)
        [sharp corners]-- cycle;
        },interior engine=empty,
    },
    sharp corners,rounded corners=southeast,arc is angular,arc=3mm,
    % The "folded paper" in the bottom right corner:
    underlay={%
        \path[fill=BgYellow!80!black] ([yshift=3mm]interior.south east)--++(-0.4,-0.1)--++(0.1,-0.2);
        \path[draw=FrameYellow,shorten <=-0.05mm,shorten >=-0.05mm,color=FrameYellow] ([yshift=3mm]interior.south east)--++(-0.4,-0.1)--++(0.1,-0.2);
        },
    drop fuzzy shadow, % Shadow
    fonttitle=\bfseries, 
    title={#1}
}
% Blue Sticky Note (BStkyNote):
\newtcolorbox{BStkyNote}[1][]{%
    enhanced,
    before skip=2mm,after skip=2mm, 
    width=0.44\textwidth, % width of the sticky note
    boxrule=0.2mm,
    colback=BgBlue, colframe=FrameBlue, % Colors
    attach boxed title to top left={xshift=0cm,yshift*=0mm-\tcboxedtitleheight},
    varwidth boxed title*=-3cm,
    % The titlebox:
    boxed title style={frame code={%
        \path[left color=FrameBlue,right color=FrameBlue,
        middle color=FrameBlue]
        ([xshift=-0mm]frame.north west) -- ([xshift=0mm]frame.north east)
        [rounded corners=0mm]-- ([xshift=0mm,yshift=0mm]frame.north east)
        -- (frame.south east) -- (frame.south west)
        -- ([xshift=0mm,yshift=0mm]frame.north west)
        [sharp corners]-- cycle;
        },interior engine=empty,
    },
    sharp corners,rounded corners=southeast,arc is angular,arc=3mm,
    % The "folded paper" in the bottom right corner:
    underlay={%
        \path[fill=BgBlue!80!black] ([yshift=3mm]interior.south east)--++(-0.4,-0.1)--++(0.1,-0.2);
        \path[draw=FrameBlue,shorten <=-0.05mm,shorten >=-0.05mm,color=FrameBlue] ([yshift=3mm]interior.south east)--++(-0.4,-0.1)--++(0.1,-0.2);
        },
    drop fuzzy shadow, % Shadow
    fonttitle=\bfseries, 
    title={#1}
}
% White Sticky Note (WStkyNote):
\newtcolorbox{WStkyNote}[1][]{%
    enhanced,
    before skip=2mm,after skip=2mm, 
    width=0.44\textwidth, % width of the sticky note
    boxrule=0.2mm,
    colback=BgWhite, colframe=FrameWhite, % Colors
    attach boxed title to top left={xshift=0cm,yshift*=0mm-\tcboxedtitleheight},
    varwidth boxed title*=-3cm,
    % The titlebox:
    boxed title style={frame code={%
        \path[left color=FrameWhite,right color=FrameWhite,
        middle color=FrameWhite]
        ([xshift=-0mm]frame.north west) -- ([xshift=0mm]frame.north east)
        [rounded corners=0mm]-- ([xshift=0mm,yshift=0mm]frame.north east)
        -- (frame.south east) -- (frame.south west)
        -- ([xshift=0mm,yshift=0mm]frame.north west)
        [sharp corners]-- cycle;
        },interior engine=empty,
    },
    sharp corners,rounded corners=southeast,arc is angular,arc=3mm,
    % The "folded paper" in the bottom right corner:
    underlay={%
        \path[fill=BgWhite!80!black] ([yshift=3mm]interior.south east)--++(-0.4,-0.1)--++(0.1,-0.2);
        \path[draw=FrameWhite,shorten <=-0.05mm,shorten >=-0.05mm,color=FrameWhite] ([yshift=3mm]interior.south east)--++(-0.4,-0.1)--++(0.1,-0.2);
        },
    drop fuzzy shadow, % Shadow
    fonttitle=\bfseries, 
    title={#1}
}

% Full Row Sticky Note (FullStkyNote):
\newtcolorbox{FullStkyNote}[1][]{%
    enhanced,
    before skip=2mm,after skip=2mm, 
    width=\textwidth, % width of the sticky note
    boxrule=0.2mm,
    colback=BgWhite, colframe=FrameWhite, % Colors
    attach boxed title to top left={xshift=0cm,yshift*=0mm-\tcboxedtitleheight},
    varwidth boxed title*=-3cm,
    % The titlebox:
    boxed title style={frame code={%
        \path[left color=FrameWhite,right color=FrameWhite,
        middle color=FrameWhite]
        ([xshift=-0mm]frame.north west) -- ([xshift=0mm]frame.north east)
        [rounded corners=0mm]-- ([xshift=0mm,yshift=0mm]frame.north east)
        -- (frame.south east) -- (frame.south west)
        -- ([xshift=0mm,yshift=0mm]frame.north west)
        [sharp corners]-- cycle;
        },interior engine=empty,
    },
    sharp corners,rounded corners=southeast,arc is angular,arc=3mm,
    % The "folded paper" in the bottom right corner:
    underlay={%
        \path[fill=BgWhite!80!black] ([yshift=3mm]interior.south east)--++(-0.4,-0.1)--++(0.1,-0.2);
        \path[draw=FrameWhite,shorten <=-0.05mm,shorten >=-0.05mm,color=FrameWhite] ([yshift=3mm]interior.south east)--++(-0.4,-0.1)--++(0.1,-0.2);
        },
    drop fuzzy shadow, % Shadow
    fonttitle=\bfseries, 
    title={#1}
}
%................................................................

\usepackage{docstyle}
\usepackage{xcolor}
\usepackage{afterpage}
\usepackage{tabularx}
\usepackage{tocloft}
\usepackage{longtable}
\usepackage{graphicx}
\usepackage{tikz}
\usepackage[parfill]{parskip}
\usepackage[final]{pdfpages}
\usepackage{float}
\usepackage{appendix}
\usepackage{listings}
\usepackage[tableposition=top]{caption}
\floatstyle{plaintop}
\restylefloat{table}
\usepackage{makecell}
\usepackage{xcolor}
\usepackage[export]{adjustbox}
\usepackage{easyfig}
\usepackage{ifthen}